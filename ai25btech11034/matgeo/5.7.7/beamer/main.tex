\documentclass{beamer}
\usepackage[utf8]{inputenc}
\usetheme{Madrid}
\usecolortheme{default}
\usepackage{amsmath,amssymb,amsfonts,amsthm}
\usepackage{txfonts}
\usepackage{tkz-euclide}
\usepackage{listings}
\usepackage{adjustbox}
\usepackage{array}
\usepackage{tabularx}
\usepackage{gvv}
\usepackage{lmodern}
\usepackage{circuitikz}
\usepackage{tikz}
\usepackage{graphicx}
\setbeamertemplate{page number in head/foot}[totalframenumber]
\usepackage{tcolorbox}
\tcbuselibrary{minted,breakable,xparse,skins}

\definecolor{bg}{gray}{0.95}
\DeclareTCBListing{mintedbox}{O{}m!O{}}{%
  breakable=true,
  listing engine=minted,
  listing only,
  minted language=#2,
  minted style=default,
  minted options={%
    linenos,
    gobble=0,
    breaklines=true,
    fontsize=\small,
    numbersep=8pt,
    #1},
  boxsep=0pt,
  left=25pt,
  right=0pt,
  top=3pt,
  bottom=3pt,
  arc=5pt,
  leftrule=0pt,
  rightrule=0pt,
  bottomrule=2pt,
  toprule=2pt,
  colback=bg,
  colframe=orange!70,
  enhanced,
  overlay={%
    \begin{tcbclipinterior}
    \fill[orange!20!white] (frame.south west) rectangle ([xshift=20pt]frame.north west);
    \end{tcbclipinterior}},
  #3,
}
\lstset{
    language=C,
    basicstyle=\ttfamily\small,
    keywordstyle=\color{blue},
    stringstyle=\color{orange},
    commentstyle=\color{green!60!black},
    numbers=left,
    numberstyle=\tiny\color{gray},
    breaklines=true,
    showstringspaces=false,
}

\title{5.7.7}
\author{AI25BTECH11034 - Sujal Chauhan}

\begin{document}

\frame{\titlepage}

\begin{frame}{Question}
If $\Vec{A}$ is a square matrix such that $\Vec{A}^2 = \Vec{A}$, then find the value of $(\Vec{I} + \Vec{A})^3 - 7\Vec{A}$.
\end{frame}

\begin{frame}{Solution}
\textbf{Solution}

Let's expand the equation:

\begin{align*}
    (\Vec{I} + \Vec{A})^3 - 7\Vec{A}
    &= (\Vec{I} + \Vec{A})^2 (\Vec{I} + \Vec{A}) - 7 \Vec{A} \\
    &= (\Vec{A}^2 + \Vec{A}\Vec{I} + \Vec{I}\Vec{A} + \Vec{I}^2) (\Vec{I} + \Vec{A}) - 7\Vec{A} \\
    &= (\Vec{A} + \Vec{A} + \Vec{A} + \Vec{I}) (\Vec{I} + \Vec{A}) - 7\Vec{A} \\
    &= (3\Vec{A} + \Vec{I}) (\Vec{I} + \Vec{A}) - 7\Vec{A} \\
    &= 3\Vec{A}\Vec{I} + 3\Vec{A}^2 + \Vec{I}^2 + \Vec{I}\Vec{A} - 7\Vec{A} \\
    &= 3\Vec{A} + 3\Vec{A} + \Vec{I} + \Vec{A} - 7\Vec{A} \\
    &= \Vec{I}
\end{align*}

\end{frame}
\end{document}

