\documentclass{beamer}
\usepackage[utf8]{inputenc}

\usetheme{Madrid}
\usecolortheme{default}
\usepackage{amsmath,amssymb,amsfonts,amsthm}
\usepackage{txfonts}
\usepackage{tkz-euclide}
\usepackage{listings}
\usepackage{adjustbox}
\usepackage{array}
\usepackage{tabularx}
\usepackage{gvv}
\usepackage{lmodern}
\usepackage{circuitikz}
\usepackage{tikz}
\usepackage{graphicx}

\setbeamertemplate{page number in head/foot}[totalframenumber]

\usepackage{tcolorbox}
\tcbuselibrary{minted,breakable,xparse,skins}



\definecolor{bg}{gray}{0.95}
\DeclareTCBListing{mintedbox}{O{}m!O{}}{%
  breakable=true,
  listing engine=minted,
  listing only,
  minted language=#2,
  minted style=default,
  minted options={%
    linenos,
    gobble=0,
    breaklines=true,
    breakafter=,,
    fontsize=\small,
    numbersep=8pt,
    #1},
  boxsep=0pt,
  left skip=0pt,
  right skip=0pt,
  left=25pt,
  right=0pt,
  top=3pt,
  bottom=3pt,
  arc=5pt,
  leftrule=0pt,
  rightrule=0pt,
  bottomrule=2pt,
  toprule=2pt,
  colback=bg,
  colframe=orange!70,
  enhanced,
  overlay={%
    \begin{tcbclipinterior}
    \fill[orange!20!white] (frame.south west) rectangle ([xshift=20pt]frame.north west);
    \end{tcbclipinterior}},
  #3,
}
\lstset{
    language=C,
    basicstyle=\ttfamily\small,
    keywordstyle=\color{blue},
    stringstyle=\color{orange},
    commentstyle=\color{green!60!black},
    numbers=left,
    numberstyle=\tiny\color{gray},
    breaklines=true,
    showstringspaces=false,
}
%------------------------------------------------------------

\title
{4.8.30}
\author 
{AI25BTECH11034 - Sujal Chauhan }



\begin{document}

\frame{\titlepage}
\begin{frame}{Question}

Find the equation of a line passing through the point (2,3,2) and parallel to the line \\
$\Vec{r}= (-2\hat{i}+3\hat{j})+\lambda(2\hat{i}-3\hat{j}+6\hat{k})$. Also, find the distance beteween these two lines.\\[2cm]
\end{frame}
\begin{frame}{Theory:}


Consider two parallel lines in 3D:
\begin{align}
    \vec{r}_1 &= \vec{a}_1 + \lambda \vec{b}, \quad \lambda \in \mathbb{R}, \\[6pt]
    \vec{r}_2 &= \vec{a}_2 + \mu \vec{b}, \quad \mu \in \mathbb{R},
\end{align}
where $\vec{a}_1, \vec{a}_2$ are points on the respective lines and $\vec{b}$ is the common direction vector.  


Shortest distance between them can be given by :
\begin{align}
    d^2 
    &= (\vec{a_2} - \vec{a_1})^\top (\vec{a_2} - \vec{a_1})
    - \left( \frac{ (\vec{a_2} - \vec{a_1})^\top \vec{b} }{ \lVert \vec{b} \rVert } \right)^2
\end{align}
Now we can calculate d.
\end{frame}

\begin{frame}{Solution:}



The direction vector of the given parallel lines is
\begin{align}
    \vec{b} = \myvec{2 \\ -3 \\ 6}.
\end{align}

The first line is given by
\begin{align}
    \vec{r}_1 = \myvec{2 \\ 3 \\ 2} + \mu \myvec{2 \\ -3 \\ 6}, \quad \mu \in \mathbb{R}.
\end{align}

The second line is
\begin{align}
    \vec{r}_2 = \myvec{-2 \\ 3 \\ 0} + \lambda \myvec{2 \\ -3 \\ 6}, \quad \lambda \in \mathbb{R}.
\end{align}
\end{frame}
\begin{frame}
Now, we compute the difference between the position vectors $\vec{a}_1$ and $\vec{a}_2$:
\begin{align}
    \vec{a}_2 - \vec{a}_1 
    = \myvec{-2 \\ 3 \\ 0} - \myvec{2 \\ 3 \\ 2} 
    = \myvec{-4 \\ 0 \\ -2}.
\end{align}

Next, we calculate the required components for the distance formula.
The squared magnitude of $(\vec{a}_2 - \vec{a}_1)$ is:
\begin{align}
    (\vec{a}_2 - \vec{a}_1)^\top (\vec{a}_2 - \vec{a}_1) &= \lVert \vec{a}_2 - \vec{a}_1 \rVert^2 \\
    &= (-4)^2 + (0)^2 + (-2)^2 \\
    &= 16 + 0 + 4 = 20.
\end{align}
\end{frame}
\begin{frame}

\begin{align}
    (\vec{a}_2 - \vec{a}_1)^\top \vec{b} &= \myvec{-4 & 0 & -2} \myvec{2 \\ -3 \\ 6} \\
    &= (-4)(2) + (0)(-3) + (-2)(6) \\
    &= -8 + 0 - 12 = -20.
\end{align}

The magnitude of the direction vector $\vec{b}$ is:
\begin{align}
    \lVert \vec{b} \rVert &= \sqrt{2^2 + (-3)^2 + 6^2} \\
    &= \sqrt{4 + 9 + 36} = \sqrt{49} = 7.
\end{align}
\end{frame}
\begin{frame}
Substituting these values into the formula for the squared distance:
\begin{align}
    d^2 
    &= (\vec{a_2} - \vec{a_1})^\top (\vec{a_2} - \vec{a_1})
    - \left( \frac{ (\vec{a_2} - \vec{a_1})^\top \vec{b} }{ \lVert \vec{b} \rVert } \right)^2 \\[6pt]
    &= 20 - \left( \frac{-20}{7} \right)^2 \\[6pt]
    &= 20 - \frac{400}{49} \\[6pt]
    &= \frac{20 \times 49 - 400}{49} \\[6pt]
    &= \frac{980 - 400}{49} \\[6pt]
    &= \frac{580}{49}.
\end{align}
\end{frame}
\begin{frame}
Finally, the shortest distance $d$ is the square root of this value:
\begin{align}
    d = \sqrt{\frac{580}{49}} = \frac{\sqrt{580}}{7} = \frac{\sqrt{4 \times 145}}{7} = \frac{2\sqrt{145}}{7}.
\end{align}
Thus, the distance between the two parallel lines is
\[
\boxed{d = \frac{2\sqrt{145}}{7}}.
\]
\end{frame}
\begin{frame}{Graph}
\begin{figure}[h]
    \centering
    \includegraphics[width=0.7\linewidth]{figures/shortest_distance.png}
    \caption{Shortest distance between two parallel lines}
    \label{fig:placeholder}
\end{figure}

jhuj

\end{frame}

\end{document}
