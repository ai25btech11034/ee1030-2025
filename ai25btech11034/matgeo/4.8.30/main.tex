\documentclass[12pt]{article}

% -------------------- Packages --------------------
\usepackage{hyperref}
\usepackage{listings}
\usepackage[margin=1in]{geometry}
\usepackage{enumitem}
\usepackage{array}
\usepackage{titlesec}
\usepackage{helvet}
\renewcommand{\familydefault}{\sfdefault}

% Math packages
\usepackage{amsmath}     % For math equations
\usepackage{amssymb}     % For advanced math symbols
\usepackage{amsfonts}    % For math fonts
\usepackage{gvv}         % Custom matrix/vector formatting
\usepackage{esint}

% Other packages
\usepackage[utf8]{inputenc}
\usepackage{graphicx}
\usepackage{pgfplots}
\pgfplotsset{compat=1.18}
\usepackage{multirow}
\usepackage{float}
\usepackage{caption}
\usepackage{multicol}

% -------------------- Formatting --------------------
\titleformat{\section}{\bfseries\large}{\thesection.}{1em}{}
\setlength{\parindent}{0pt}
\setlength{\parskip}{6pt}
\renewcommand{\labelenumi}{\alph{enumi})}

% -------------------- Document --------------------
\begin{document}

\newpage
\begin{center}
\textbf{\Large AI25BTECH11034 - SUJAL CHAUHAN }\\
\textbf{4.8.30}
\end{center}

\textbf{Question:}\\
Find the equation of a line passing through the point (2,3,2) and parallel to the line \\
$\Vec{r}= (-2\hat{i}+3\hat{j})+\lambda(2\hat{i}-3\hat{j}+6\hat{k})$. Also, find the distance beteween these two lines.\\[2cm]

\textbf{Theory:}  

Consider two parallel lines in 3D:
\begin{align}
    \vec{r}_1 &= \vec{a}_1 + \lambda \vec{b}, \quad \lambda \in \mathbb{R}, \\[6pt]
    \vec{r}_2 &= \vec{a}_2 + \mu \vec{b}, \quad \mu \in \mathbb{R},
\end{align}
where $\vec{a}_1, \vec{a}_2$ are points on the respective lines and $\vec{b}$ is the common direction vector.  


Shortest distance between them can be given by :
\begin{align}
    d^2 
    &= (\vec{a_2} - \vec{a_1})^\top (\vec{a_2} - \vec{a_1})
    - \left( \frac{ (\vec{a_2} - \vec{a_1})^\top \vec{b} }{ \lVert \vec{b} \rVert } \right)^2
\end{align}
Now we can calculate d.



\textbf{Solution:}\\[2cm]
The direction vector of the given parallel lines is
\begin{align}
    \vec{b} = \myvec{2 \\ -3 \\ 6}.
\end{align}

The first line is given by
\begin{align}
    \vec{r}_1 = \myvec{2 \\ 3 \\ 2} + \mu \myvec{2 \\ -3 \\ 6}, \quad \mu \in \mathbb{R}.
\end{align}

The second line is
\begin{align}
    \vec{r}_2 = \myvec{-2 \\ 3 \\ 0} + \lambda \myvec{2 \\ -3 \\ 6}, \quad \lambda \in \mathbb{R}.
\end{align}

Now, we compute the difference between the position vectors $\vec{a}_1$ and $\vec{a}_2$:
\begin{align}
    \vec{a}_2 - \vec{a}_1 
    = \myvec{-2 \\ 3 \\ 0} - \myvec{2 \\ 3 \\ 2} 
    = \myvec{-4 \\ 0 \\ -2}.
\end{align}

Next, we calculate the required components for the distance formula.
The squared magnitude of $(\vec{a}_2 - \vec{a}_1)$ is:
\begin{align}
    (\vec{a}_2 - \vec{a}_1)^\top (\vec{a}_2 - \vec{a}_1) &= \lVert \vec{a}_2 - \vec{a}_1 \rVert^2 \\
    &= (-4)^2 + (0)^2 + (-2)^2 \\
    &= 16 + 0 + 4 = 20.
\end{align}


\begin{align}
    (\vec{a}_2 - \vec{a}_1)^\top \vec{b} &= \myvec{-4 & 0 & -2} \myvec{2 \\ -3 \\ 6} \\
    &= (-4)(2) + (0)(-3) + (-2)(6) \\
    &= -8 + 0 - 12 = -20.
\end{align}

The magnitude of the direction vector $\vec{b}$ is:
\begin{align}
    \lVert \vec{b} \rVert &= \sqrt{2^2 + (-3)^2 + 6^2} \\
    &= \sqrt{4 + 9 + 36} = \sqrt{49} = 7.
\end{align}

Substituting these values into the formula for the squared distance:
\begin{align}
    d^2 
    &= (\vec{a_2} - \vec{a_1})^\top (\vec{a_2} - \vec{a_1})
    - \left( \frac{ (\vec{a_2} - \vec{a_1})^\top \vec{b} }{ \lVert \vec{b} \rVert } \right)^2 \\[6pt]
    &= 20 - \left( \frac{-20}{7} \right)^2 \\[6pt]
    &= 20 - \frac{400}{49} \\[6pt]
    &= \frac{20 \times 49 - 400}{49} \\[6pt]
    &= \frac{980 - 400}{49} \\[6pt]
    &= \frac{580}{49}.
\end{align}

Finally, the shortest distance $d$ is the square root of this value:
\begin{align}
    d = \sqrt{\frac{580}{49}} = \frac{\sqrt{580}}{7} = \frac{\sqrt{4 \times 145}}{7} = \frac{2\sqrt{145}}{7}.
\end{align}
Thus, the distance between the two parallel lines is
\[
\boxed{d = \frac{2\sqrt{145}}{7}}.
\]
\newpage
\begin{figure}[h]
    \centering
    \includegraphics[width=0.7\linewidth]{figures/shortest_distance.png}
    \caption{Shortest distance between two parallel lines}
    \label{fig:placeholder}
\end{figure}

\end{document}
